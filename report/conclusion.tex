% !TEX root = report.tex
\section{Conclusion}
We have seen that implementing elimination for a shared stack with queue delegation overall gives better performance when compared with a simple non-elimination shared stack with queue delegation. We have seen the effect on performance due to delegation queue size, elimination array size, number of threads and percentage of push operations. Apart from a few data anomalies as discussed in the Results / Analysis section, the rest of the results matched our expectations.

Several areas of this project can be further explored in the future. First, we have found that the implementation style of the QD Lock libraries was very disruptive to updating for elimination. Future work can focus on making the libraries more amenable to this type of customization, so that other features can also be included (such as an efficient elimination array range policy). Second, further experiments could be performed with MonitorT if it is updated to include elimination. This would allow more comparison with the QD Elimination Lock implementation. Last, the data collected here represents a broad, coarse spectrum of configurations. More work can be done to focus on the best-performing configurations, leading to exploration of better and more fine-grained configurations.
