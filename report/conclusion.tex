% !TEX root = report.tex
\section{Conclusion}
In conclusion, we have seen that implementing elimination for a shared stack overall gives better performance when compared with a simple non-elimination queue delegation. We have seen the effect on performance due to delegation queue size, elimination array size, number of threads and percentage of push operations. Apart from a few data anomalies as discussed in the Analysis section, the rest of the results matched our expectations.

Further experiments could be performed with MonitorT also implementing elimination. Additionally, the data collected represents a broad, coarse spectrum of configurations. More work can be done to focus on the best performing configurations and perform experiments with more fine-grained configurations.

We found that the implementation style of the QD Lock libraries was very disruptive to updating for elimination. Future work can focus on making the libraries more amenable to this type of work, so that other features can also be included (such as an efficient range policy).
