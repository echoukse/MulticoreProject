% !TEX root = report.tex
\section{Results / Analysis}
We recognize that the Java and C++ queue delegation implementations will have inherent performance differences. The purpose of the analysis in this section is to determine a baseline performance difference between MonitorT (Java) and QD Lock (C++), where neither have elimination. Then, we analyze the performance difference by adding elimination to the QD Lock version (i.e., QD Elimination Lock).

Figure~\ref{fig:baseline} shows the baseline performance comparison between MonitorT and QD Lock (neither having stack elimination). Figure~\ref{fig:fig00} shows that the two implementations are relatively similar, with the C++ implementation slightly outperforming MonitorT, as would be expected. It is interesting to note that they both seem to scale equally well with increasing thread counts, with the exception of 8 threads (which can be attributed to the contention inherent with running 8 threads on an 8-threaded machine). However, Figure~\ref{fig:fig01} shows that their performance is relatively volatile when the delegation queue size varies. As the queue size increases, both implementations tend to have longer run times.

\begin{figure}[]
\centering
\subfloat[][]{\label{fig:fig00}\includegraphics[width=.49\textwidth]{figs/00_TimeVsThreads_cppNoElim_javaNoElim.eps}}
\subfloat[][]{\label{fig:fig01}\includegraphics[width=.49\textwidth]{figs/01_TimeVsQDsize_cppNoElim_javaNoElim.eps}}\\
\caption[]{Baseline comparison of the MonitorT implementation (Java) and the QD Lock (C++), without elimination for either implementation: \subref{fig:fig00} varying the number of threads (queue size = 256, push ratio = 0.5); \subref{fig:fig01} varying the delegation queue size (threads = 4, push ratio = 0.5).}
\label{fig:baseline}
\end{figure}

%======================================================================

In Figure~\ref{fig:fig02}, we see that the performance gap between MonitorT and QD Elimination Lock is greater than the gap between MonitorT and QD Lock in Figure~\ref{fig:fig00}. This expected behavior also confirms that the performance of QD Elimination Lock is better than QD Lock. The performance benefits are greater at higher thread counts. After this initial comparison of MonitorT, QD Lock and QD Elimination Lock, more focus can be given to analyzing the performance benefits of moving from QD Lock to QD Elimination Lock.

\begin{figure}[!h]
\centering
\includegraphics[width=.75\textwidth]{figs/02_TimeVsThreads_cppElim_javaNoElim.eps}
\caption[]{Comparison of the MonitorT implementation (Java) and the QD Elimination Lock (C++), varying the number of threads (elimination included for C++ implementation, not for Java). Queue size = 256, elimination array size = 4, push ratio = 0.5.}
\label{fig:fig02}
\end{figure}

%======================================================================

Figure~\ref{fig:qdlock} shows that QD Elimination Lock consistently outperforms QD Lock, across sweeps of all parameters. Additional insights can also be gained from these figures. Figure~\ref{fig:fig03} shows that as the number of threads increases, the speedup of QD Elimination Lock over QD Lock increases. Figure~\ref{fig:fig04} shows that choosing the correct elimination array size can be very important. The elimination array size should be neither too small nor too large, as this yields either too much contention (shown where array size = 2) or too few matches between pushes and pops (shown where array size = 16). In Figure~\ref{fig:fig05}, as the delegation queue size increases, the delegate thread spends more time executing the other threads' workloads, eventually slowing the execution of the delegate thread's own work. This is seen for both the QD Lock and QD Elimination Lock. This behavior would not be expected in MonitorT, since MonitorT uses a separate thread for the delegation queue. In Figure~\ref{fig:fig06}, as the push percentage increases, we would have expected the performance to decrease since there would be fewer matches between pushes and pops. But instead we get the unexpected result that the performance either increases or remains steady. Another anomaly is that we would expect QD Lock to have flat performance across all push percentages, since it views all operations alike.

\begin{figure}[]
\centering
\subfloat[][]{\label{fig:fig03}\includegraphics[width=.49\textwidth]{figs/03_TimeVsThreads_cppElim_cppNoElim.eps}}
\subfloat[][]{\label{fig:fig04}\includegraphics[width=.49\textwidth]{figs/04_TimeVsElsize_cppElim_cppNoElim.eps}}\\
\subfloat[][]{\label{fig:fig05}\includegraphics[width=.49\textwidth]{figs/05_TimeVsQDsize_cppElim_cppNoElim.eps}}
\subfloat[][]{\label{fig:fig06}\includegraphics[width=.49\textwidth]{figs/06_TimeVsPctPush_cppElim_cppNoElim.eps}}
\caption[]{QD Elimination Lock versus QD Lock (both C++), while varying each parameter separately: \subref{fig:fig03} varying the number of threads (elimination array size = 4, queue size = 256, push ratio = 0.5); \subref{fig:fig04} varying the elimination array size (threads = 4, queue size = 256, push ratio = 0.5); \subref{fig:fig05} varying the delegation queue size (threads = 4, elimination array size = 4, push ratio = 0.5); \subref{fig:fig06} varying the push ratio (threads = 4, elimination array size = 4, queue size = 256).}
\label{fig:qdlock}
\end{figure}

%======================================================================
\clearpage
The performance values observed in Figure~\ref{fig:thrd_and_elsize} show the relationships between thread count, delegation queue size and elimination array size. Figures~\ref{fig:fig07a} --~\ref{fig:fig07d} represent changing the delegation queue size between 128,  256, 512 and 1024, respectively. The different queue sizes appear to have a negligible effect on the run time, except for the first case where the performance is better with a queue size of 128. These figures also show that the thread count had the largest effect and the elimination array size had minimal effect, since the multiple lines plotted for run time versus thread count exhibit minimal changes for the different elimination array sizes.

\begin{figure}[!t]
\centering
\subfloat[][]{\label{fig:fig07a}\includegraphics[width=.49\textwidth]{figs/07a_Q128_TimeVsThreadsVsEsize_cppElim.eps}}
\subfloat[][]{\label{fig:fig07b}\includegraphics[width=.49\textwidth]{figs/07b_Q256_TimeVsThreadsVsEsize_cppElim.eps}}\\
\subfloat[][]{\label{fig:fig07c}\includegraphics[width=.49\textwidth]{figs/07c_Q512_TimeVsThreadsVsEsize_cppElim.eps}}
\subfloat[][]{\label{fig:fig07d}\includegraphics[width=.49\textwidth]{figs/07d_Q1024_TimeVsThreadsVsEsize_cppElim.eps}}
\caption[]{QD Elimination Lock performance versus thread count and elimination array size - each graph represents a different delegation queue size, while multiple elimination array sizes are plotted in each graph (with push ratio = 0.5): \subref{fig:fig07a} queue size = 128; \subref{fig:fig07b} queue size = 256; \subref{fig:fig07c} queue size = 512; \subref{fig:fig07d} queue size = 1024.}
\label{fig:thrd_and_elsize}
\end{figure}

%======================================================================
The elimination array size's minimal effect on performance is reflected again in Figure~\ref{fig:fig08b}, where the performance for each thread count remains relatively flat across all elimination array sizes. This plot also shows how increasing the number of threads monotonically decreases the performance (except for 8 threads), establishing the effects of contention at the higher thread counts. The difference in behavior for the line depicting 8 threads is likely due to contention from running 8 threads on a 8-threaded machine. Figure~\ref{fig:fig08a} shows that initially, increasing the delegation queue size adversely affects performance. However, for delegation queue sizes above 1024, the performance levels out.

\begin{figure}[]
\centering
\subfloat[][]{\label{fig:fig08b}\includegraphics[width=.49\textwidth]{figs/08b_Q128_TimeVsEsizeVsThreads_cppElim.eps}}
\subfloat[][]{\label{fig:fig08a}\includegraphics[width=.49\textwidth]{figs/08a_E4_TimeVsEsizeVsThreads_cppElim.eps}}
\caption[]{\subref{fig:fig08b} QD Elimination Lock performance versus elimination array size and thread count (queue size = 256, push ratio = 0.5). \subref{fig:fig08a} QD Elimination Lock performance versus delegation queue size and thread count (elimination array size = 4, push ratio = 0.5).}
\label{fig:qdsize_and_thrd}
\end{figure}
