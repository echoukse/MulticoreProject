% !TEX root = report.tex
\section{Results / Analysis}
We recognize that the Java and C++ queue delegation implementations will have inherent performance differences. The purpose of the analysis in this section is to determine a baseline performance difference between MonitorT (Java) and QD Lock (C++), where neither have elimination. Then, we analyze the performance difference by adding elimination to the QD Lock version (i.e., QD Elimination Lock).

Figure~\ref{fig:baseline} shows the baseline performance comparison between MonitorT and QD Lock (neither having stack elimination). Figure~\ref{fig:fig00} shows that the two implementations are relatively similar, with the C++ implementation slightly outperforming MonitorT, as would be expected. It is interesting to note that they both seem to scale equally well with increasing thread counts. However, Figure~\ref{fig:fig01} shows that their performance is relatively volatile when the delegation queue size varies. As the queue size increases, both implementations have longer run times, but the MonitorT implementation generally loses performance at a faster rate, but it does exhibit an unusual peak in performance for a delegation queue size of 2048.

\begin{figure}[]
\centering
\subfloat[][]{\label{fig:fig00}\includegraphics[width=.49\textwidth]{figs/00_TimeVsThreads_cppNoElim_javaNoElim.eps}}
\subfloat[][]{\label{fig:fig01}\includegraphics[width=.49\textwidth]{figs/01_TimeVsQDsize_cppNoElim_javaNoElim.eps}}\\
\caption[]{Baseline comparison of the MonitorT implementation (Java) and the QD Lock (C++): \subref{fig:fig00} varying the number of threads, and without elimination for either implementation (queue size = 256, push ratio = 0.5); \subref{fig:fig01} varying the delegation queue size, and without elimination for either implementation (threads = 4, push ratio = 0.5).}
\label{fig:baseline}
\end{figure}

%======================================================================

In Figure~\ref{fig:fig02}, we see that the performance of QD Elimination Lock is better than QD Lock in Figure~\ref{fig:fig00}, as expected. The performance benefits are greater at higher thread counts.

\begin{figure}[]
\centering
\includegraphics[width=.75\textwidth]{figs/02_TimeVsThreads_cppElim_javaNoElim.eps}
\caption[]{Comparison of the MonitorT implementation (Java) and the QD Elimination Lock (C++), varying the number of threads (elimination included for C++ implementation, not for Java). Queue size = 256, elimination array size = 4, push ratio = 0.5.}
\label{fig:fig02}
\end{figure}

%======================================================================



\begin{figure}[]
\centering
\subfloat[][]{\label{fig:fig03}\includegraphics[width=.49\textwidth]{figs/03_TimeVsThreads_cppElim_cppNoElim.eps}}
\subfloat[][]{\label{fig:fig04}\includegraphics[width=.49\textwidth]{figs/04_TimeVsElsize_cppElim_cppNoElim.eps}}\\
\subfloat[][]{\label{fig:fig05}\includegraphics[width=.49\textwidth]{figs/05_TimeVsQDsize_cppElim_cppNoElim.eps}}
\subfloat[][]{\label{fig:fig06}\includegraphics[width=.49\textwidth]{figs/06_TimeVsPctPush_cppElim_cppNoElim.eps}}
\caption[]{QD Elimination Lock versus QD Lock (both C++), while varying each parameter separately: \subref{fig:fig03} varying the number of threads (elimination array size = 4, queue size = 256, push ratio = 0.5); \subref{fig:fig04} varying the elimination array size (threads = 4, queue size = 256, push ratio = 0.5); \subref{fig:fig05} varying the delegation queue size (threads = 4, elimination array size = 4,push ratio = 0.5); \subref{fig:fig06} varying the push ratio (threads = 4, elimination array size = 4, queue size = 256).}
\label{fig:qdlock}
\end{figure}

%======================================================================
The performance values observed in Figure~\ref{fig:thrd_and_elsize} also show the relationships between thread count, delegation queue size and elimination array size. Figures~\ref{fig:fig07a} --~\ref{fig:fig07d} represent changing the delegation queue size from 128 to 2048, respectively. The different queue sizes appear to have a negligible effect on the run time, except for the first case where the performance is better with a queue size of 128. These figures also show that the thread count had the largest effect and the elimination array size had minimal effect, since the multiple lines plotted for run time versus thread count exhibit minimal changes for the different elimination array sizes.

\begin{figure}[]
\centering
\subfloat[][]{\label{fig:fig07a}\includegraphics[width=.49\textwidth]{figs/07a_Q128_TimeVsThreadsVsEsize_cppElim.eps}}
\subfloat[][]{\label{fig:fig07b}\includegraphics[width=.49\textwidth]{figs/07b_Q256_TimeVsThreadsVsEsize_cppElim.eps}}\\
\subfloat[][]{\label{fig:fig07c}\includegraphics[width=.49\textwidth]{figs/07c_Q512_TimeVsThreadsVsEsize_cppElim.eps}}
\subfloat[][]{\label{fig:fig07d}\includegraphics[width=.49\textwidth]{figs/07d_Q1024_TimeVsThreadsVsEsize_cppElim.eps}}
\caption[]{QD Elimination Lock performance versus thread count and elimination array size - each graph represents a different delegation queue size, while multiple elimination array sizes are plotted in each graph: \subref{fig:fig07a} queue size = 128; \subref{fig:fig07b} queue size = 256; \subref{fig:fig07c} queue size = 512; \subref{fig:fig07d} queue size = 1024. Push ratio = 0.5.}
\label{fig:thrd_and_elsize}
\end{figure}

%======================================================================
The elimination array size's minimal effect on performance is reflected again in Figure~\ref{fig:08b}, where the performance for each thread count remains relatively flat across all elimination array sizes. This plot also shows how increasing the number of threads monotonically decreases the performance, establishing the effects of contention at the higher thread counts.

Figure~\ref{fig:fig08a} shows that initially, increasing the delegation queue size adversely affects performance. However, for delegation queue sizes above 1024, the performance levels out.

\begin{figure}[]
\centering
\subfloat[][]{\label{fig:fig08b}\includegraphics[width=.49\textwidth]{figs/08b_Q128_TimeVsEsizeVsThreads_cppElim.eps}}
\subfloat[][]{\label{fig:fig08a}\includegraphics[width=.49\textwidth]{figs/08a_E4_TimeVsEsizeVsThreads_cppElim.eps}}
\caption[]{\subref{fig:fig08b} QD Elimination Lock performance versus elimination array size and thread count. Push ratio = 0.5. \subref{fig:fig08a} QD Elimination Lock performance versus delegation queue size and thread count.}
\label{fig:qdsize_and_thrd}
\end{figure}
